\indent
С каждым днем пользователи сети интернет пропускают через себя всё больший и больший поток информации, и оставить только полезную её часть помогают современные рекомендательные системы. Такие системы активно применяются в социальных сетях, принося пользу и администрациям сайтов, и пользователям. С одной стороны, пользователю предоставляются более релевантные выдача информации и показ постов в новостной ленте. С другой стороны, заинтересовывая и удерживая пользователей, владельцы социальной сети получают большие выплаты от рекламодателей.

\indent
Классификация и кластеризация информации – это те задачи, которые могут возникнуть в процессе построения рекомендательной системы. Например, имея обученный классификатор, можно автоматически понижать приоритет постов в новостной выдаче, если их тематика не соответствует тематике публикующего пост сообщества. Другой пример использования обученного классификатора – рекомендация пользователю похожего контента.
