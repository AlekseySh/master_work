\section*{Введение}

\indent
\indent
С каждым днем пользователи социальных сетей создают и потребляют все б\'{о}льшие
и б\'{о}льшие объемы информации, в том числе огромное количество фотографий.
Построение системы для быстрой и точной навигации в миллионах изображений
 --- не тривиальная задача. Одно из самых распространенных решений этой
 проблемы заключается в использовании тегов, в частности
 использование хештегов в социальных сетях.


\indent
\textit{Хештег} --- это любое слово или фраза без пробелов, перед которой стоит 
символ \#, который называется \textit{диез} или \textit{решетка}, а в англоязычном 
варианте -- \textit{hash}, отсюда и название. Приведем несколько примеров:
\htag{masterwork}, \htag{spbu}, \#htag{lovesport}. Обычно в браузерах или 
приложениях хештеги отображаются как гипертекст, кликнув по которому можно 
получить список публикаций, снабженных таким же или похожими хэштегами.


\indent
Кроме простоты и удобства использования теги обладают еще одним полезным свойством 
-- они позволяют не думать об  иерархии структурируемой информации. 
Например, набор изображений можно разложить
по папкам, создав иерархию по датам, геолокациям или авторству. Причем в отдельных
случаях подобрать наиболее подходящую иерархию бывает затруднительно. Проблему
можно решить так: достаточно поставить по несколько тегов для всех изображений,
которые могут храниться в плоской системе файлов.
Благодаря этому свойству тегирование используются  для рубрикации контента 
не только только онлайн, но и в оффлайн приложениях, например, 
просмоторщиках фотографий. 


\indent 
В настоящей работе будет описано построение и обучение модели, подсказывающей 
пользователям социальных сетей популярные хэштеги, релевантные к загружаемым
изображениям. Для количественного измерения корректности работы системы 
кроме расчёта формальных метрик будет приведена точность,
оценённая пользователями вручную.


\indent
Кроме того, с помощью разрабатываемой системы потенциально можно решать и 
обратную задачу -- определять,
уместно ли поставлены те или иные теги к заданным изображениям. 
Способность системы давать ответ на такой вопрос можно использовать для 
выявления злоупотреблений со стороны пользователей. Например, зачастую
 в рекламных целях продвигаемую публикацию снабжают множеством популярных 
тегов, не имеющих никакого отношения к её содержимому.
 
 