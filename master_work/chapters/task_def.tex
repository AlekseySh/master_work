\section{Постановка задачи}

todo Рассказать про другие работы о тегах (например, на фликре).

\indent
\indent
Данная работа посвящена разработке интеллектуальной системы, 
подсказывающей пользователю релевантные и широко распространенные
хештеги к загружаемым фотографиям.
Можно выделить две разновидности популярных хэштегов в социальных сетях.
Первые являются выражением чувств, эмоций и других абстрактных понятий,
например \htag{love}, \htag{friendship}, \htag{happy}. Вторые непосредственно
связаны с объектами, находящимися в кадре, например  
\htag{sport}, \htag{cafe}, \htag{nature}. 
Интуитивно понятно, что для второй разновидности
тегов предсказательная система, построенная на машинном зрении, будет
давать более точные результаты, чем для первой (но будет
иметь относительно узкую специализацию). Чтобы построить такую систему
нужен подходящий набор обучающих данных, но существующие датасеты, 
собранные из социальных сетей, не разделяют тренировочные примеры на те, 
где визуальный контент напрямую связан с целевыми классами, и те, где это не так.
В связи с чем возникла идея для данной задачи 
адаптировать датасет, изначально не размеченный
хэштегами, но содержащий аннотации, связанные с объектами
 в кадре. Кроме того, датасет должен быть относительно небольшим,
чтобы не требовать значительных 
вычислительных ресурсов; но при этом достаточно разнообразным, чтобы 
быть релевантным как можно большему количеству загружаемого контента.
В настоящей работе автор остановил свой выбор на датасете мест и локаций 
 \textit{SUN\cite{sundata} (Scene Understanding Dataset)}, 
 разметка которого после некоторой предобработки может быть отображена в популярные хэштеги.


\indent  
\textit{SUN} -- это набор фотографий, для каждой из которых выбрана одно
 из четырехсот названий локаций (сцен), вот несколько примеров:
  
  
\begin{itemize}
    \item \textit{baseball field}
    \item \textit{basketball court}
    \item \textit{ice shelf}
    \item \textit{forest}
    \item \textit{wind farm}
\end{itemize}


\indent
Большинство названий локаций сами по себе не являются популярными 
тегами из социальных сетей. Поэтому необходимо
сопоставить их широко распространенным хэштегам (если это возможно):
  
  \begin{itemize}
      \item \textit{baseball field, basketball court}  $\rightarrow$ \htag{sport}
      \item \textit{ice shelf, forest} $\rightarrow$ \htag{nature}
      \item \textit{wind farm} $\rightarrow \times$ 
  \end{itemize}
  
\indent
Обученная на таких данных модель по окружению,
обнаруженному на пользовательском фото,
сможет подсказывать ему подходящий хэштег.



\indent
\indent
Ясно, что полученная модель будет корректно работать лишь для
ограниченного (пусть и большого) домена классов. Следовательно, необходимо 
обучить её распознавать не входящие в этот домен изображения и
не пытаться определить их категорию. Кроме того, в случае низкой уверенности
в правильности предсказания так же лучше ничего не делать. 
По мнению автора, гораздо предпочтительнее не 
предложить пользователю подходящий хэштег, чем многократно предлагать 
не релевантные варианты.
 
   
\indent
\indent
В качестве моделей компьютерного зрения в работе 
используются глубокие сверточные
нейронные сети. Данный выбор обосновывается тем, что в последние
годы сверточные сети отлично зарекомендовали себя для решения задач, связанных
с обработкой изображений. Mожно привести в пример самый большой 
и известный
конкурс по классификации изображений \textit{ImageNet}\cite{imagenet},
который проводится ежегодно с 2010 года, при этом, начиная с 2012 года, классические
решения ни разу не побеждали, уступив место нейронным сетям.


\indent
\indent
Научная новизна работы определяется адаптацией датасета \textit{SUN} для решения задачи о структурировании изображений в социальных сетях.
