\section*{Заключение}


\indent
\indent
\indent
В ходе проделанной работы был реализован прототип системы,
предлагающей пользователям социальных сетей популярные 
хэштеги к загружаемым изображениям.


\bigskip
\indent
\indent
Вместо использования классических для данной задачи наборов данных,
был специальным образом адаптирован 
датасет для распознавания сцен и локаций \textit{SUN}.
По логике автора, это позволит модели, обученной на таких данных,
давать более точные предсказания для хэштегов, связанных
с  физическими сущностями (такими как \htag{nature}, \htag{sport}, \htag{sky}).
Улучшение точности достигается за счет того, что предсказание модели всегда связано
с тем, какие физические объекты находятся в кадре. С другой стороны,
такой подход не позволяет предлагать хэштеги, отражающие чувства, эмоции
или другие абстрактные понятия (такие как \htag{love}, \htag{friendship}, \htag{happy}).


\bigskip
\indent
\indent
Главным компонентом разработанной системы является свёрточная 
нейронная сеть. По результатам множества численных экспериментов
были выбраны гиперпараметры для обучения модели, 
соответствующие максимальному 
значению целевой метрики на тестовом подмножестве данных.
Были исследованы паттерны возникновения ошибочных предсказаний.


\bigskip
\indent
\indent
Чтобы понять, насколько предложения модели адекватны с точки зрения 
человека, а не формальных метрик, был собран ещё один набор изображений
из социальных сетей. Для него с были сделаны 
предсказания, затем проверенные вручную. 
По мнению проверяющих, более чем в 80\% случаев 
подсказанный хэштег является уместным.
