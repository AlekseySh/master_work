\section{Искуственные нейронные сети}

\indent
\indent
Настоящая часть работы предназначена для читателя, не знакомого с 
искуственными нейронными сетями. Здесь будут приведены теоретические 
основы глубокого обучения и рассмотрены сверточные архитектуры, 
используемые в дальнейшей работе.

\subsection{Базовая информация о глубоком обучении}

\indent
\indent
Начнем с рассмотрения одиночного нейрона
 --- перцептрона Розенблатта --- базового элемента, содержащегося в большинстве современных нейросетевых архитектур.
Перцептрон имеет несколько входов и один выход, значение на котором
вычисляется как взвешенная сумма значений входов \ref{eq:perceptron}.
Кроме того, обычно
к выходному значению применяется сдвиг и некоторая нелинейная функция, 
называющаяся функцией активации нейрона. Ее предназначение мы обсудим позже.

todo pic

\begin{equation}\label{eq:perceptron}
    f(\vec{x}) = S(\sum_{i=1}^n x_i w_i + b)
\end{equation}

где $f(\vec{x})$ -- выходное значения нейрона, посчитанное для входов $x_i$,
$w_i$ -- весовые коэффициенты для входов, $b$ -- параметр смещения, 
а $g$ --- нелинейная функция активации.

\indent
\indent
Существуют множество различных функций активации, например, гиперболический
тангенс, логистическая сигмоида или \textit{ReLU} \ref{eq:activations}. Перечисленные
функции особенно популярны, так как значения их производных простым образом 
выражаются через значения самой функции, что, как будет показано далее, позволяет
ускорить процесс обучения нейросети.


\begin{equation}\label{eq:activations}
	\begin{gathered}
	    S(x) = th(x) = \frac{e^x - e^{-x}}{e^x + e^-x},    \;   th’(x) = 1 - th(x)^2   \\    
	    S(x) = \sigma(x) = \frac{1}{1 + e^-x},   \;   \sigma’(x) = \sigma(x)(1 - \sigma(x)) \\
	    S(x) = ReLU = max(0, x),   \;   ReLU’(x) = \theta(x)
	\end{gathered}
\end{equation}
где $\theta(x)$ -- функция Хэвисайда.


\indent
По своей сути перцептрон является линейной моделью классификации. 
 
 Объединение нейронов в слои.
 Сведение к матрицам
 Вычислительный граф
 Обучение обратным распостранением ошибки.
 
 
\subsection{Сверточные нейронные сети}
Свертка
Пулинг
Сверточная сеть


\subsection{Используемые архитектуры}
В данное работе в качестве базовой неросетевой архитектуры используется
ResNet \cite{resnet}.
