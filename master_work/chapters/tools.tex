\section{Используемые инструменты}

\indent
\indent
\textbf{Программные средства}

\indent
В качестве языка программирования использовался
 \textit{Python 3.6.7} (сборка \textit{Anaconda}),
 в качестве среды разработки --- \textit{PyCharm Professional 2018.1},
 операционная система \textit{Ubuntu 16.04.4 LT}. Использованы
 следующие сторонние библиотеки для \textit{python}:


\begin{itemize}

    \item \textit{pytorch, torchvision} --- построение и обучение нейронных сетей
    \item \textit{tensorboardX} --- визуальное логирование процесса обучения 
    \item \textit{PIL, opencv, scikit-image, scipy} --- обработка изображений
    \item \textit{matplotlib} --- отрисовка графиков
    \item \textit{numpy} --- матричные вычисления
    \item \textit{pandas} --- работа с таблицами
    \item \textit{nltk} --- работа с текстом, в том числе с базой \textit{WordNet}
    \item \textit{scikit-learn} --- библиотека машинного обучения общего назначения
    \item \textit{pip} --- пакетный мененджер
    \item \textit{InstaLooter} --- утилита для автоматического скачивания
    изображений и видео из сети \textit{Instagram} по заданному хэштегу.
   
\end{itemize}


\bigskip

\indent
\indent
\textbf{Вычислительные мощности}

\indent
\indent
Обучение моделей производилось на удаленном сервере со следующей конфигурацией:
\begin{itemize}
    \item видеокарта \textit{GEFORCE GTX 1080 Ti} (\textit{11 ГБ} видеопамяти)
    \item процессор \textit{AMD Ryzen Threadripper 1920X 12-Core}
    \item оперативная память объемом  \textit{64 ГБ}
\end{itemize}
